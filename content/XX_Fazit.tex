\newpage

\section{Fazit}\label{sec:fazit}

Zwischen vielen Geräten wird zur Kommunikation der Standard HTTP,
bzw. bei der Kommunikation nach Außen sogar HTTPS verwendet.

Erfreulicherweiße ist die Kommunikation mit dem Webserver im Internet damit durch TLS geschützt.

Die interne Kokmmunikation mit der emulierten \textit{Hue Brige} hingegen ist nicht geschützt.
Weder Integrität noch Vertraulichkeit sind damit gesichert.
Somit kann ein Angreifer, der sich im internen Netzwerk befindet,
die gesendeten Pakete passiv mitlesen.
Dadurch erfährt er,
welche Geräte im Netzwerk über den \textit{Home Assistant} und die emulierte \textit{Hue Brige} gesteuert werden.

Weitaus schlimmer ist jedoch, dass er die GET- und POST-Requests auch selbst stellen kann.
Zum einen muss er dann nicht (ggf. aufwändig) das Netzwerk belauschen
und zum anderen kann er auf diesem Weg auch Geräte selbst steuern.

Für Privatanwender ist dies eine potentielle Bedrohung,
Allerdings ist davon auszugehen,
dass ein Angreifer im internen Netz aufgrund der überschaubaren Größe eher unwahrscheinlich ist.
Außerdem hält sich in diesem Beispiel der Schaden in Grenzen, da nur Unterhaltungsgeräte ein- bzw. ausgeschaltet werden.
Würde man auf gleiche Art und Weise ein Türschloss oder ähnliches kontrollieren erhöht das jedoch durch einen
möglichen Einbruch den potentiellen Schaden.

Im Gegensatz zu den anderen Übertragungen wird
in der TCP-Verbindung zwischen \textit{Home Assistant} und Playstation 4 kein allgemein bekannter Standard verwendet.
Die Verbindung ist zwar nicht über TLS, dafür aber auf Anwendungsebene verschlüsselt.
Hierdurch ist die Vertraulichkeit der Verbindung gesichert.
Zusätzlich werden jedoch ungesichert Pakete mittels UDP übertragen.

Zusammenfassend ist anzumerken,
dass ein großer Teil der Kommunikation über bekannte Standards bewältigt wird.
Zzumindest die Verbindungen, welche über das öffentliche Netz übertragen werden, sind gesichert.
Über die unverschlüsselten lokalen Verbindungen kann im Privatgebrauch in einem sicheren Netz hinweggesehen werden.
Sollte jedoch auch hier eine gesicherte Übertragung erforderlich sein,
so ist von dem hier gezeigten Aufbau abzuraten.