\section{Aufbau und Konfiguration}\label{sec:aufbau}
Die in Kapitel \ref{sec:komponenten} \textit{\nameref{sec:komponenten}} beschriebenen Komponenten werden über verschiedene Technologien miteinander verküpft.
In \autoref{fig:aufbau} wird grundlegend dargestellt welche Kompoennten miteinander verbunden sind.
Hierbei ist anzumerken, dass Verbindungen über (W)LAN indirekt über einen Router erstellt werden.
Zur Wahrung der Übersichtlichkeit sind lediglich die abstrakten Verbindungen auf Anwendungsebene dargestellt.

\begin{figure}[ht!]
    \centering
    \resizebox{\textwidth}{!}{
        \input{assets/uml/Aufbau.latex}
    }
    \caption{Aufbau}
    \label{fig:aufbau}
\end{figure}

\subsection{Harmony Hub}\label{sec:aufbau-hub}
Im Zentrum des Schaubilds steht der Harmony Hub.
Er selbst ist über WLAN mit dem Heimnetzwerk verbunden.
Angesteuert wird der Harmony Hub per LAN entweder von der auf dem Smartphone installierten App oder vom Echo Dot.

Der Hub ist an den ersten Raspberry Pi (gennant \textit{Mediacenter}, da auf diesem Kodi installiert ist)
und an die Playstation 4 über Bluetooth als Fernbedienung angeschlossen.
In beiden Fällen ist hierzu das bei Bluetooth-Geräten übliche \enquote{Pairing} notwendig.

Außerdem ist de Hub an den zweiten Raspberry Pi (gennant \textit{Hassbian}, da dort das Betriebssystem genutzt wird) per LAN angeschlossen.
Diese besondere Schnittstelle wird später im Detail betrachtet.

\subsection{Raspberry Pi - Hassbian}\label{sec:aufbau-hassbian}
Auf diesem Raspberry Pi ist das Betriebssytem \textit{Hassbian} installiert.
Das darin enthaltene Home Assistant emuliert die sogennante \enquote{Hue Bridge} \cite{Emulated83:online}.
Eigentlich ist die Hue Brige zentraler Bestandteil des smarten Lampensystems der Marke Phillips \cite{HueBridg65:online}.
In Home Assistant wird die emulierte Bridge genutzt, um eine Verbindung zum Harmony Hub herzustellen.
Home Assistant stellt hierzu sogennante \textit{Schalter} bereit, welche über die \textit{Hue API} an den Harmony Hub präsentiert werden.
Aus Sicht des Hubs kommuniziert er mit einer \enquote{normalen} Hue Bridge.
Jedoch lassen sich über die Schalter mehr als nur Lichter steuern.
Eine mächtige Möglichkeit ist das in diesem Projekt genutzte ausführen von Kommandozeilenbefehlen über eine entsprechende Schnittstelle (\ac{cli}).

Das \ac{cli} wird genutzt, um die Playstation 4 unter Nutzung des Programms \textit{ps4-waker}\cite{dhleongp12:online} zu steuern.
Doch wieso wird ein solch komplexer Aufbau benötigt, wenn der Harmony Hub bereits eine direkte Verbindung zur Playstation 4 besitzt?
Die Playstation lässt sich zwar per Bluetooth steuern, allerdings lässt sie sich auf diesem Weg nicht einschalten.
Das Programm \textit{ps4-waker} hingegen erlaubt es die \ac{ps4} ein- sowie auszuschalten und sogar bestimmte Anwndungen zu starten.
\todo[inline]{Details für PS4-waker? z.B. Prüfung, ob PS4 ein}

\subsection{Raspberry Pi - Mediacenter}\label{sec:aufbau-mediacenter}
Dieser Raspberry Pi wird durch \textit{Raspbian} betrieben.
Hierbei ist das Programm \textit{Kodi} als Anwendung installiert.
Insgesamt wird das Programm über zwei Schnittstellen gesteuert.

Einerseits bietet das Programm selbst eine LAN-Schnittstelle.
In diesem Projekt wird diese über die Smartphone-App \textit{Kore} genutzt.

Andererseits kann \textit{Kodi} auch von Geräten gesteuert werden, welche an den Raspberry Pi direkt angeschlossen sind.
Dies kann eine klassische Tastatur oder (wie in diese Projekt) über Bluetooth angeschlossenes Gerät sein.
Der Harmony Hub funktioniert hierbei ähnlich wie eine Bluetooth-Tastatur.