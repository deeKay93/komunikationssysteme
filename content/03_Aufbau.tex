\newpage
\section{Aufbau und Konfiguration}\label{sec:aufbau}
Die in Kapitel \ref{sec:komponenten} \textit{\nameref{sec:komponenten}} beschriebenen Komponenten werden über verschiedene Technologien miteinander verküpft.
In \autoref{fig:aufbau} wird grundlegend dargestellt welche Kompoennten miteinander verbunden sind.
Hierbei ist anzumerken, dass Verbindungen über (W)LAN indirekt über einen Router erstellt werden.
Zur Wahrung der Übersichtlichkeit sind lediglich die abstrakten Verbindungen auf Anwendungsebene dargestellt.

\begin{figure}[ht!]
    \centering
    \resizebox{\textwidth}{!}{
        \input{assets/uml/Aufbau.latex}
    }
    \caption{Aufbau}
    \label{fig:aufbau}
\end{figure}

\subsection{Harmony Hub}\label{sec:aufbau-hub}
Zentrales Element ist der Harmony Hub.
Er selbst ist über WLAN mit dem Heimnetzwerk verbunden.
Angesteuert wird der Harmony Hub per LAN von der auf dem Smartphone installierten App.

Der Hub ist an die Playstation 4 über Bluetooth als Fernbedienung angeschlossen.
Hierzu ist im Vorfeld das bei Bluetooth-Geräten übliche \enquote{Pairing} notwendig.

Außerdem ist de Hub an den Raspberry Pi per LAN angeschlossen.
Diese besondere Schnittstelle wird im Foglgenden detailiert betrachtet.

\subsection{Raspberry Pi / Hassbian}\label{sec:aufbau-hassbian}
Auf dem Raspberry Pi ist das Betriebssytem \textit{Hassbian} installiert.
Das darin enthaltene Home Assistant emuliert die sogennante \enquote{Hue Bridge} \cite{Emulated83:online}.
Eigentlich ist die Hue Brige zentraler Bestandteil des smarten Lampensystems der Marke Phillips \cite{HueBridg65:online}.
In Home Assistant wird die emulierte Bridge genutzt, um eine Verbindung zum Harmony Hub herzustellen.
Home Assistant stellt hierzu sogennante \textit{Schalter} bereit, welche über die \textit{Hue API} an den Harmony Hub als Lichter präsentiert werden.
Aus Sicht des Hubs kommuniziert er mit einer \enquote{normalen} Hue Bridge.
Jedoch lassen sich über die Schalter mehr als nur Lichter steuern.
Eine weitere Möglichkeit ist das in diesem Projekt genutzte Ausführen von Kommandozeilenbefehlen über eine entsprechende Schnittstelle (\ac{cli}).

Das \ac{cli} wird genutzt, um die Playstation 4 unter Nutzung des Programms \textit{ps4-waker}\cite{dhleongp12:online} zu steuern.
Doch wieso wird ein solch komplexer Aufbau benötigt, wenn der Harmony Hub bereits eine direkte Verbindung zur Playstation 4 besitzt?
Die Playstation lässt sich zwar per Bluetooth steuern, allerdings lässt sie sich auf diesem Weg nicht einschalten.
Das Programm \textit{ps4-waker} hingegen erlaubt es die \ac{ps4} ein- sowie auszuschalten und sogar bestimmte Anwndungen zu starten.
Das Einschalten geschieht über einen Befehl der Form \lstinline[language=bash]{ps4-waker -d 192.168.178.34  -c ~/.homeassistant/.ps4-wake.credentials.json}.
\lstinline[language=bash]{-d} gibt hierbei die IP-Adresse des Gerätes und
\lstinline[language=bash]{-c} die Datei,
welche die notwendigen Zugangsdaten enthält, an.
Durch anhängen des Schlüsselworts \lstinline[language=bash]{standby} kann die
\ac{ps4} wieder ausgeschaltet werden.
Die Zugangsdaten werden zuvor durch folgendes Pairing-Verfahren ermittelt:
\begin{enumerate}
    \setlength\itemsep{-0.5em}
    \item \textit{ps4-waker} gibt sich beim ersten Start als Playstation 4 aus.
    \item Mit einem Smartphone versucht man sich über die entsprechende App mit der falschen \ac{ps4} zu verbinden.
    \item \textit{ps4-waker} nutzt die so übertragenen Informationen um sich gegenüber der echten \ac{ps4} als Smartphone-App auszugeben.
    \item Die so erhaltenen Zugangsdaten werden in einer JSON-Datei gespeichert.
\end{enumerate}


Der Home Assistant wird so konfiguriert,
dass die Befehle des \textit{ps4-waker} über einen Schalter ausgeführt werden können.
In \autoref{lst:ps4_switch} ist die Konfiguration des Schalters dargestellt,
welcher zum Ein- und Ausschalten der Playstation 4 dient.

\lstinputlisting[
    caption=Konfiguration des Schalters,
    label=lst:ps4_switch,
    language=yaml
]{ps4_switch.yaml}

Die erste Zeile gibt die eindeutige ID für den Schalter an.
Durch den \lstinline[language=yaml]{friendly_name} kann zusätzlich ein nutzerfreundlicher Name zur Anzeige auf Benutzeroberflächen angegeben werden.

Der aktuelle Zustand des Schalters wird mit \lstinline[language=yaml]{command_state} und
\lstinline[language=yaml]{value_template} ermittelt.
Der Kommandozeilenbefehl unter \lstinline[language=yaml]{command_state} wird regelmäßig ausgeführt.
Als Ergebnis liefert er die in \autoref{lst:ps4-waker_search} dargestellten Daten im JSON-Format.
In \lstinline[language=yaml]{value_template} wird auf dieses JSON über
\lstinline[language=yaml]{'value_json'} zugegriffen.
Genauer gesagt wird geprüft,
ob das Feld \lstinline[language=json]{"statusCode"} den Wert
\lstinline[language=json]{"200"} enthält.
Ist dies der Fall, so gilt der Schalter als eingeschaltet, andernfalls als ausgeschaltet.

\lstinputlisting[
    caption=Antwoort von \textit{ps4-waker search}-Befehl (gekürzt),
    label=lst:ps4-waker_search,
    language=json
]{ps4-waker_search.json}

Je nach aktuellem Status des Schalters führt ein betättigen dazu, dass der Kommandozeilenbefehl unter \lstinline[language=yaml]{command_on} zum einschalten
oder der unter \lstinline[language=yaml]{command_off} zum Ausschalten ausgeführt wird.
Beides wird über die jeweiligen Befehle von \textit{ps4-waker} umgesetzt.